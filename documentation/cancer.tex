\section{El cáncer}

El cáncer es el nombre genérico usado para referirse a más de 200 enfermedades cuyo resultado
es la proliferación descontrolada de las células que se ven afectadas. Aunque no era común
encontrar signos de cáncer en los seres vivos en la antiguedad debido a una menor esperanza
media de vida, lo cierto es que es inherente al reino animal.

El origen etimológico del nombre proviene de la antigua grecia, donde un médico llamado
Hipocrates utilizaba los términos \textit{carcino} y \textit{carcinoma} para referirse
a dicha enfermedad, y las cuales hacian referencia al cangrejo debido a la forma en la
que se proyectaban los cánceres.

El cáncer se clasifica como:

\begin{itemize}
    \item Una \textbf{enfermedad genética}, debido a, que es causada por un cambio en el
    material genético (ADN) de las células.
    \item Una \textbf{enfermedad multigénica}, ya que, afecta a diferentes genes.
    \item Una \textbf{enfermedad multifactorial}, es decir, tiene causas muy diversas.
    \item Una \textbf{enfermedad multiorgánica}, por tanto, afecta a diferentes tejidos y organos.
\end{itemize}

Se conoce como neoplasia a toda masa anormal que tenga lugar en el cuerpo. Esto es, la división
descontrolada de la célula y el aumento de tamaño de cada una de ellas, y ocurre
por un problema en el ADN de la célula, la cual, contiene la información acerca de cuales son
las funciones de la misma.

Al producirse en todo tipo de tejidos la evolución y el tratamiento de esta enfermedad varía.
El factor determinante entre que este tipo de masas anormales sea benigna (adenoma)
o maligna (carcinoma) es el factor de crecimiento y la invasividad de tejido adyacente.

Presenta diferentes fases, partiendo de un crecimiento anormal en tamaño, en la cual,
cada vez necesitará más nutrientes. Esa necesidad de nutrientes hace que la célula entre en
una segunda fase conocida como \textit{angiogénesis}, en la cual, crea nuevos vasos sanguíneos
para lograr su objetivo. Finalmente, la célula se vasculariza, es decir, partes de ella pasan
a la sangre, continuando con este comportamiento allí donde se depositen y, además, en muchas
ocasiones se observa metástasis.

Es obvio, que la célula para llegar a la última fase va a necesitar cierto grado de daño genético,
así como, superar determinadas barreras. Por un lado, obtener todas las capacidades necesarias para
poder dividirse sin control, crear sus pripios vasos sanguíneos, etc. dependerá de la obtención
de mutaciones en su ADN que permitan todo esto. Por otro lado, el cuerpo tiene mecanismos para evitar
que esto ocurra, como por ejemplo, muerte por daño genético, evitar enviar orden de división e, incluso,
el propio sistema inmune interviene para intentar evitar su crecimiento entre otros.

Por tanto, esta es una enfermedad compleja de tipo ecológico y emergente, lo cual, dificulta su estudio,
así como, su tratamiento.
