El cáncer es un nombre genérico que agrupa más de 200 enfermedades\footnote{\url{https://www.aacrfoundation.org/Pages/what-is-cancer.aspx}}
que causan proliferación descontrolada de células provocando la aparición de masas anormales. A toda masa anormal se
le conoce como neoplasia y, según su invasividad y factor de crecimiento, puede ser maligna (carcinoma) o benigna (adenoma).

Esta enfermedad puede ser vista como un comportamiento emergente, que puede ser explicado a
partir de la presencia de ciertos marcadores cancerosos a nivel local.

José Santos y Ángel Monteagudo, autores de una serie de artículos \cite{jsantos-amonteagudo-2012} \cite{jsantos-amonteagudo-2013} \cite{jsantos-amonteagudo-2015} y, entre ellos, el artículo
en el que se centra este trabajo \cite{jsantos-amonteagudo-1-2014}, proponen el uso de autómatas
celulares como técnica para simular dicho comportamiento dada su idoneidad.

Se propone una modelización en la cual se utiliza una rejilla en tres dimensiones donde, sin
tener en cuenta el tamaño de la célula, se presenta al comienzo una única célula en el centro de dicha
rejilla. Cada célula cuenta con su propio genoma, conjunto de genes que las distingue, y que provoca
diferentes comportamientos entre ellas. Además, cuenta con dos propiedades adicionales:
tamaño del telómero, y tasa de mutación base. El tamaño del telómero supone un límite al
número de veces que una célula puede realizar la división celular, ya que este se ve acortado cada vez.
La tasa de mutación base define la probabilidad de una célula de adquirir una nueva mutación
en su genoma al realizar la división celular.

Todo ello, es utilizado dentro de un modelo de eventos, en el cual, se programan eventos mitóticos
en el futuro y en el que se consideran una serie de pruebas previas a la mitosis. Dichas pruebas,
modelizan el comportamiento individual de cada célula a la hora de ejecutar la división, así como,
todos aquellos factores que pueden derivar en la muerte de la célula y factores provocados
por las mutaciones adquiridas.

Finalmente, se realizan una serie de simulaciones, alterando los parámetros de configuración del
autómata, con el objetivo de estudiar el nuevo comportamiento, así como, la incidencia
de determinadas terapias en el comportamiento global del sistema.
