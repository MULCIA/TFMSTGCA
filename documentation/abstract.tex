El cáncer es un nombre genérico que agrupa más de 200 enfermedades que causan proliferación
descontrolada de células provocando la aparición de masas anormales. A toda masa anormal se
le conoce como neoplasia y, según su inavisividad y factor de crecimiento, puede ser maligna
(carcinoma) o benigna (adenoma).

Esta enfermedad puede ser vista como un comportamiendo emergente, que puede ser explicado a
partir de la presencia de ciertos marcadores cancerosos a nivel local.

José Santos y Ángel Monteagudo, autores de una serie de artículos y, entre ellos, el artículo
en el que se centra este trabajo, proponen el uso de autómatas celulares como técnica para
simular dicho comportamiendo dada su similitud e idoneidad.

Se propone una modelización en el cual se utiliza una rejilla en tres dimensiones donde, sin
tener en cuenta el tamaño de la célula, se presenta una única célula en el centro de dicha
rejilla. Cada célula, cuenta con un genoma asociado que representa la aparición de mutaciones,
así como, dos propiedades: tamaño del telomero, y tasa de mutación base.

Todo ello, es utilizado dentro de un modelo de eventos, en el cual, se programan eventos mitóticos
en el futuro y en el que se consideran una serie de pruebas previas a la mitosis. Dichas pruebas,
modelizan el comportamiendo individual de cada célula a la hora de ejecutar la división, así como,
todos aquellos factores que pueden derivar en la muerte de la célula y factores provocados
por las mutaciones adquiridas.

Finalmente, se realizan una serie de simulaciones, utilizando una serie de parámetros globales
concretos en cada caso, para estudiar la evolución y el comporamiento del sistema.
