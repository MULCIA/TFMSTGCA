José Santos y Ángel Monteagudo realizaron una serie de simulaciones en su artículo
donde intentan estudiar la influencia de los marcadores en la evolución de la simulación.
Todas las simulaciones, y sus resultados, se comentan en esta sección.

El tamaño de rejilla utilizada para todos los experimentos es de $50^5$. Esto son,
$125.000$ células posibles en la rejilla.

\section{Influencia del parámetro \textit{Tasa de mutación base (m)}}

Los autores en su artículo presentan 3 experimentos utilizando los valores por defecto
y variando el parámetro \textit{Tasa de mutación base (m)} para estudiar como afecta
dicho parámetro en la proliferación del cáncer.

Cada experimento se muestra en una serie de gráficas en las cuales se presenta
como resultado la media de 5 ejecuciones diferentes.

A continuación, se muestra cada uno de los experimentos especificando en cada caso
el valor $m$ utilizado. El resto de parámetros de la simulación se mantiene constante
con los valores considerados por defectos, que son:

\begin{table}[h!]
  \centering
  \caption{Valores de los parámetros, excepto \textit{m}.}
  \label{tab:table1}
  \begin{tabular}{ccc}
    \toprule
    Nombre & Símbolo & Valor\\
    \midrule
    Tamaño del telómero & tl & 50\\
    Muerte por daño genético & e & 10\\
    Factor de incremento de tasa de mutación base & i & 100\\
    Muerte de un vecino & g & 30\\
    Muerte aleatoria & a & 1000\\
    \bottomrule
  \end{tabular}
\end{table}

\subsection{Experimento 1: Tasa de mutación base igual a 10.000}

Pruebas.

\subsection{Experimento 2: Tasa de mutación base igual a 1.000}

Pruebas.

\subsection{Experimento 3: Tasa de mutación base igual a 100}

Pruebas.

\section{Influencia del resto de marcadores}

Para observar el comportamiento del resto de marcadores, los autores, realizan una variación
de los parámetros respecto al experimento anterior. Los valores utilizados son los siguientes:

\begin{table}[h!]
  \centering
  \caption{Valores de los parámetros.}
  \label{tab:table1}
  \begin{tabular}{ccc}
    \toprule
    Nombre & Símbolo & Valor\\
    \midrule
    Tasa de mutación base & m & 100.000\\
    Tamaño del telómero & tl & 35\\
    Muerte por daño genético & e & 20\\
    Factor de incremento de tasa de mutación base & i & 100\\
    Muerte de un vecino & g & 10\\
    Muerte aleatoria & a & 400\\
    \bottomrule
  \end{tabular}
\end{table}

Prueba.
