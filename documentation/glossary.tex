\begin{description}
    \item[\textbf{Adenoma}] Masa anormal que tiene un comportamiento benigno, es decir, un crecimiento leve y una invasividad de tejido adyacente bajo.
    \item[\textbf{ADN}] Abreviación de ácido desoxirribonucleico, consiste en un ácido nucléico que contiene las instrucciones genéticas usadas en el desarrollos y funcionamiento de todos los organismos vivos.
    \item[\textbf{Apoptosis}] Mecanismo de muerte celular programada para controlar y frenar el crecimiento de células que tienen su código genético dañado y, en consecuencia, pueden provocar problemas en el organismo.
    \item[\textbf{Cáncer}] Nombre genérico para describir a las enfermedades que causa proliferación descontrolada de células que provocan la aparición de masas anormales.
    \item[\textbf{Carcinoma}] Masa anormal que tiene un comportmaiento maligno, es decir, tiene una tasa de crecimiento e invasividad alta.
    \item[\textbf{Genoma}] Conjunto de genes contenidos en el cromosomas.
    \item[\textbf{Mitosis}] Proceso de reproducción de una célula, mediante el cual, se crea una copia exacta de la misma. Puede contener daños genéticos que no se dan en la célula original.
    \item[\textbf{Singleton}] Lorem ipsum dolor sit amet, consectetur adipisicing elit, sed do eiusmod tempor incididunt ut labore et dolore magna aliqua.
    \item[\textbf{Telómero}] Lorem ipsum dolor sit amet, consectetur adipisicing elit, sed do eiusmod tempor incididunt ut labore et dolore magna aliqua.
\end{description}
