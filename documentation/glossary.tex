\begin{description}
    \item[\textbf{Adenoma}] Masa anormal que tiene un comportamiento benigno, es decir, un crecimiento leve y una invasividad de tejido adyacente bajo.
    \item[\textbf{ADN}] Abreviación de ácido desoxirribonucleico, consiste en un ácido nucleico que contiene las instrucciones genéticas usadas en el desarrollos y funcionamiento de todos los organismos vivos.
    \item[\textbf{Angiogénesis}] Proceso mediante el cual se forman nuevos vasos sanguíneos nuevos a partir de los preexistentes.
    \item[\textbf{Apoptosis}] Mecanismo de muerte celular programada para controlar y frenar el crecimiento de células que tienen su código genético dañado y, en consecuencia, pueden provocar problemas en el organismo.
    \item[\textbf{Autómata Celular}] Modelo matemático para un sistema dinámico que evoluciona en pasos discretos en base a unas reglas locales.
    \item[\textbf{Cáncer}] Nombre genérico para describir a las enfermedades que causa proliferación descontrolada de células que provocan la aparición de masas anormales.
    \item[\textbf{Carcinoma}] Masa anormal que tiene un comportamiento maligno, es decir, tiene una tasa de crecimiento e invasividad alta.
    \item[\textbf{Genoma}] Conjunto de genes contenidos en el cromosomas.
    \item[\textbf{Metástasis}] Reproducción o extensión de una enfermedad o de un tumor a otra parte del cuerpo.
    \item[\textbf{Mitosis}] Proceso de reproducción de una célula, mediante el cual, se crea una copia exacta de la misma. Puede contener daños genéticos que no se dan en la célula original.
    \item[\textbf{Mutación}] Alteración repentina y permanente de la estructura genética o cromosómica de la célula de un ser vivo que se transmite a sus descendientes por herencia.
    \item[\textbf{Neoplasia}] Masa anormal que aparece en alguna parte del cuerpo por el crecimiento descontrolado de células.
    \item[\textbf{Singleton}] Patrón de diseño en el cual el software tiene un punto único de entrada, o una sóla instancia de la aplicación.
    \item[\textbf{Telómero}] Son los extremos de los cromosomas, regiones del \textit{ADN} no codificante, cuyo objetivo es proteger la parte codificante para evitar que sufra daños durante el proceso de replicación.
\end{description}
