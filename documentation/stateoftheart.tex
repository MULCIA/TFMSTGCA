Para cumplir con uno de los primeros objetivos de este trabajo, se necesitaba
realizar una búsqueda de artículos científicos sobre esta temática:
\textit{Simulación de crecimiento de tumores con autómatas celulares}.

Tras una primera selección, donde se descartaron los artículos de \textit{arXiv}\footnote{\url{https://arxiv.org/}},
ya que, podrían estar aún en fase de revisión, así como, artículos demasiado antiguos o artículos
cuyo enfoque quedaran fuera del ámbito de la inteligencia artificial.

De los artículos seleccionados, se procedió a su estudio y extracción de información. A continuación,
se presentan y describen los trabajos candidatos.

\section{\textit{Cellular automaton of idealized brain tumor growth dynamics}}

Este artículo \cite{kansal-torquato} es uno de los dos candidatos estudiados con el objetivo
de desarrollar este proyecto.

Se presenta un nuevo modelo de autómata celular para simular crecimientos emergente de tumores cerebrales.
En concreto, se enfoca a un tipo de tumor, el tumor de Gompertzian. Los autores presentan
la consecución de simular el crecimiento de dicho tumor en casi tres órdenes de magnitud en radio
utilizando sólo cuatro parámetros microscópicos.

Completar.

\section{\textit{Analysis of behaviour transitions in tumour growth
using a cellular automaton simulation}}

Este artículo \cite{jsantos-amonteagudo-1-2014} forma parte de una serie de artículos
\cite{jsantos-amonteagudo-2012} \cite{jsantos-amonteagudo-2013} \cite{jsantos-amonteagudo-2015}
en el cual los autores, con un enfoque genérico, pretenden simular el crecimiento de
tumores.

Para ello, se basan en varios trabajos, pero principalmente en los trabajos de Douglas Hanahan y Robert A. Weinberg
\cite{hanahan-weinberg-2000} \cite{hanahan-weinberg-2011}, donde se presentan varios marcadores presentes
en las células, los cuales consisten en una serie de mutaciones que permiten a las células presentar
un comportamiento canceroso.

En su enfoque, utilizan un autómata celular que sigue un modelo de eventos, esto es, porque los autores
necesitan, por un lado, simular las escalas de tiempo, y por otro, modelizar cuándo las células necesitan
realizar la mitosis.

Completar.
