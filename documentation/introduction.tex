Desde el punto de vista de la biología computacional el cáncer puede ser visto como un sistema ecológico y de comportamiento emergente.
Lo cual, hace apropiado el uso de autómatas celulares para simular su comportamiento. Es por ello que, cada vez más se opta por el uso
de simulaciones por ordenador como la que se presenta en este trabajo ayudando, desde el estudio y comprensión del cáncer, hasta
el estudio y desarrollo de nuevas terapias contra él.

El cáncer es una enfermedad cada vez más común debido a sus muy diversos factores que propician su aparición, esto es,
mayor esperanza de vida de la población y, por tanto, individuos cuyas células han sufrido más mutaciones, factores ambientales y/o
formas de vida poco saludables, mayor estrés de la población, y hasta exposición a fuentes de radiación artificiales.

A pesar de todo, el cuerpo tiene una serie de mecanismos para prevenir la aparición y la proliferación de esta enfermedad. Estos,
son muchos y muy diversos, así como, los mecanismos propios de las células que intervienen en este proceso. En esta simulación, por tanto,
se pretenden tener en cuenta el enfoque de los autores del trabajo original, eligiendo sólo un subconjunto de todas estas
carácteristicas y comportamientos.

La elección de un autómata celular como enfoque para modelizar el comportamiento de esta enfermedad proviene por la relativa
facilidad de obtener comportamientos complejos a partir de una reglas relativamente sencillas a nivel local en cada célula.

Este trabajo, como se ha comentado previamente, pretende reproducir el trabajo de José Santos y Ángel Monteagudo de Julio de 2014,
uno de los muchos trabajos que dichos autores han realizado siguiendo la idea de realizar simulaciones sobre esta enfermedad y, por ello, aquí
se realiza una nueva implementación con el fin de reproducir los resultados obtenidos en el citado artículo.

La implementación consiste en seguir un modelo de eventos, es decir, se pretende simular el proceso que siguen las células para su división, que
consiste en, la realización de una serie de pruebas para comprobar si la célula debe reproducirse, la realización de la propia división y,
la programación de un nuevo evento de división en el futuro. A lo largo de la ejecución del programa en cada iteración puede haber ninguna, una o
varias células que deben seguir el proceso que se acaba de describir brevemente.

Finalmente, para estudiar el comportamiento del cáncer se realizan diversas ejecuciones utilizando diferentes parámetros y configuraciones,
y observando en cada caso qué factor es el que provoca el cambio en el comportamiento global.

\section{Objetivos}

En este trabajo, se esperan alcanzar los siguientes objetivos:

\begin{itemize}
  \item Encontrar artículos cientificos de calidad sobre esta temática.
  \item Extraer información del artículo (o los artículos).
  \item Elegir un artículo y hacer un modelado e implementación propia.
  \item Reproducir los resultados del artículo.
  \item Comparar los resultados obtenidos con los del artículo.
\end{itemize}

\section{Alcance}

El presente trabajo tiene como finalidad reproducir el trabajo de José Santos y Ángel Monteagudo de Julio de 2014, realizando
una implementación propia del sistema que se describe en el presente documento.

No obstante, el comportamiento que se pretende simular cuenta con una serie de pasos en los cuales el comportamiendo dependerá
de ciertos sorteos con una probabilidad dada. Esta es una limitación a tener en cuenta a la hora de obtener los mismos resultados
que los autores de dicho trabajo.

\section{El cáncer}

El cáncer es el nombre genérico usado para referirse a más de 200 enfermedades cuyo resultado
es la proliferación descontrolada de las células que se ven afectadas. Aunque no era común
encontrar signos de cáncer en los seres vivos en la antiguedad debido a una menor esperanza
media de vida, lo cierto es que es inherente al reino animal.

El origen etimológico del nombre proviene de la antigua grecia, donde un médico llamado
Hipocrates utilizaba los términos \textit{carcino} y \textit{carcinoma} para referirse
a dicha enfermedad, y las cuales hacian referencia al cangrejo debido a la forma en la
que se proyectaban los cánceres.

El cáncer se clasifica como:

\begin{itemize}
    \item Una \textbf{enfermedad genética}, debido a, que es causada por un cambio en el
    material genético (ADN) de las células.
    \item Una \textbf{enfermedad multigénica}, ya que, afecta a diferentes genes.
    \item Una \textbf{enfermedad multifactorial}, es decir, tiene causas muy diversas.
    \item Una \textbf{enfermedad multiorgánica}, por tanto, afecta a diferentes tejidos y organos.
\end{itemize}

Se conoce como neoplasia a toda masa anormal que tenga lugar en el cuerpo. Esto es, la división
descontrolada de la célula y el aumento de tamaño de cada una de ellas, y ocurre
por un problema en el ADN de la célula, la cual, contiene la información acerca de cuales son
las funciones de la misma.

Al producirse en todo tipo de tejidos la evolución y el tratamiento de esta enfermedad varía.
El factor determinante entre que este tipo de masas anormales sea benigna (adenoma)
o maligna (carcinoma) es el factor de crecimiento y la invasividad de tejido adyacente.

Presenta diferentes fases, partiendo de un crecimiento anormal en tamaño, en la cual,
cada vez necesitará más nutrientes. Esa necesidad de nutrientes hace que la célula entre en
una segunda fase conocida como \textit{angiogénesis}, en la cual, crea nuevos vasos sanguíneos
para lograr su objetivo. Finalmente, la célula se vasculariza, es decir, partes de ella pasan
a la sangre, continuando con este comportamiento allí donde se depositen y, además, en muchas
ocasiones se observa metástasis.

Es obvio, que la célula para llegar a la última fase va a necesitar cierto grado de daño genético,
así como, superar determinadas barreras. Por un lado, obtener todas las capacidades necesarias para
poder dividirse sin control, crear sus pripios vasos sanguíneos, etc. dependerá de la obtención
de mutaciones en su ADN que permitan todo esto. Por otro lado, el cuerpo tiene mecanismos para evitar
que esto ocurra, como por ejemplo, muerte por daño genético, evitar enviar orden de división e, incluso,
el propio sistema inmune interviene para intentar evitar su crecimiento entre otros.

Por tanto, esta es una enfermedad compleja de tipo ecológico y emergente, lo cual, dificulta su estudio,
así como, su tratamiento.

\section{Los autómatas celulares}

Un autómata celular es una de las formas más adecuadas de simular comportamientos emergentes,
al igual que los sistemas de Lindenmayer y, por tanto, adecuado para este tipo de problemas.
José Santos y Ángel Monteagudo optaron por utilizar autómatas celulares para sus trabajos.

Consiste en modelo matemático que modela sistemas dinámicos. Está inspirado en cómo ciertas especias
han resuelto el problema de qué color o forma tomar en su aspecto exterior, como por ejemplo, es
el caso de ciertas conchas marinas. Este mecanismo, al ser favorable desde el punto de vista evolutivo,
les ha hecho sobrevivir gracias a una mejor mimetización con su entorno.

Este modelo necesita de unas propiedades, tales como, dimensión, estados posibles de sus celdas,
qué se considera vecindario de una célula y, por último, cuales son las reglas de evolución. Además,
hay que definir cómo es su frontera.

Es, por tanto, muy adecuado para ser implementado en ordenadores.
