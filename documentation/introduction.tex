\section{Introducción}
	Podemos pensar en nuevas formas de evaluar a los alumnos de una clase de manera más interactiva a la que estamos acostumbrados a ver en nuestras aulas. Una de las opciones que se podría elegir consiste en crear una comunidad donde los alumnos se ayuden entre sí, bien utilizando foros, sistemas de preguntas y respuestas (stackoverflow, yahoo! answers, etc) o wikis.

	Pero esto no es nada nuevo. Ya tenemos asignaturas que usan foros o wikis para ayudar a sus alumnos con el temario, que se impliquen en la asignatura ampliando contenidos explicados en clase, etcétera.

	Este tipo de comunidades tienen algunos problemas que hacen que muchos profesores terminen desestimándolas como parte de la evaluación, ya que a la hora de evaluar es necesario cerciorar el contenido y su calidad, si es que no se ha hecho antes; intentar purgar contenido repetido; recopilar todo lo ofrecido por cada alumno y evaluar de un modo objetivo todo este material.

	Éste es un trabajo bastante arduo y más cuando no es la única asignatura a la que se dedica un profesor. Además, también resulta complicado por parte de los alumnos el seguimiento de estos métodos.

	En los primeros cursos, sobre todo, los alumnos no se conocen entre ellos y es difícil crear un clima de cooperación. Otras veces se topan con plataformas complejas y terminan usando otros métodos de cooperación, como por ejemplo compartir la información en persona. Y en algunas ocasiones, la recompensa por los esfuerzos realizados por los alumnos que se interesan en buscar o desarrollar material cae en la incertidumbre. Éstas serían algunas de las causas por las que los estudiantes terminan frustrados y desechando estas comunidades.

	Para obtener una solución a este problema es interesante buscar sistemas de motivación para que los grupos funcionen de forma dinámica y fluida. Una de las opciones, y en la que se centra este proyecto, podría ser lo que en inglés se conoce por \emph{gamification}. El recompensar a los usuarios el trabajo desempeñado através de sistemas de puntos e insignias en un ambiente lúdico puede incidir positivamente en la colaboración y el aprendizaje.

	Es por ello que este proyecto tiene como meta la creación de un sistema de insignias para una plataforma de enseñanza y colaboración, de manera que los profesores puedan evaluar según unas métricas determinadas el esfuerzo de los alumnos. Además, éstos pueden mostrar sus logros, usarlos o integrarlos en otros sistemas.

	Por supuesto, uno de los fines de otorgar insignias a los usuarios por conseguir sus metas es poder mostrar estos distintivos a los demás. Uno de los sistemas creados para abordar esta problemática es \textbf{Mozilla Open Badges}, que permite a cualquier entidad emitir insignias a una serie de ganadores y también permite a éstos tener un lugar donde almacenarlas y mostrarlas de forma selectiva.

	Ya que \textbf{Mozilla Open Badges} sólo es una infraestructura para el control de las insignias, se necesita una plataforma donde los usuarios puedan ganarlas. Como plataforma se ha elegido \textbf{Askbot}, un proyecto de código abierto escrito en Python con el framework web \textbf{Django}. Este sistema permite a los usuarios cooperar para encontrar la mejor respuesta a sus preguntas. Mediante un sistema de votos, positivos o negativos, karma e insignias la comunidad de usuarios valoran las preguntas y respuestas creadas en la aplicación. Los administradores, moderadores o usuarios de peso en la comunidad, pueden completar o modificar las preguntas y respuestas. Creando así un ecosistema favorable para el aprendizaje.

	\textbf{Nota:} Hay que tener en cuenta que \textbf{Django} hace una implementación particular del patrón Modelo-Vista-Controlador. Lo define como Modelo-Plantilla-Vista (MTV: model-template-view), donde la vista equivale al controlador y la plantilla a la vista. Dado que este proyecto está creado en \textbf{Django} se usará esta termininología en la documentantación.

\section{Alcance}
	Este sistema tendrá como usuarios a los alumnos y profesores de la asignatura DAD, impartida por el profesor Pablo Trinidad del departamento de Lenguajes y Sistemas informático de la Universidad de Sevilla.

	El Objetivo es facilitar la evaluación a los profesores de la asignatura mediante el desarrollo de una aplicación.

	Aunque la aplicación nace para la evaluación de alumnos en una asignatura concreta, puede ser apliada a cualquier otro campo.

\section{Objetivos}
	Con el desarrollo de \textbf{OBadges} se esperan alcanzar estos objetivos:
	\begin{itemize}
		\item Dotar de un sistema evaluación complementario mediante un sistema lúdico de logros, frente a los habituales sistemas de evaluación que la mayoría de veces pueden ser inviables para profesores y alumnos.
		\item Crear una herramienta con la que dar insignias a los alumnos que hayan asistido a actividades de la comunidad universitaria\footnote{Talleres, cursos o actividades extraescolares.}.
		\item Ofrecer a los alumnos una forma de compartir sus logros en la universidad, conseguidos en una asignatura o actividad, mediante la integración con \textbf{Mozilla Open Badges}.
	\end{itemize}
