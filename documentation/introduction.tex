\section{Introducción}

		Desde el punto de vista de la biología computacional el cáncer puede ser visto como un sistema ecológico y de comportamiento emergente.
		Lo cual, hace apropiado el uso de autómatas celulares para simular su comportamiento. Es por ello que, cada vez más se opta por el uso
		de simulaciones por ordenador como la que se presenta en este trabajo ayudando, desde el estudio y comprensión del cáncer, hasta
		el estudio y desarrollo de nuevas terapias contra él.

		El cáncer es una enfermedad cada vez más común debido a sus muy diversos factores que propician su aparición, esto es,
		mayor esperanza de vida de la población y, por tanto, individuos cuyas células han sufrido más mutaciones, factores ambientales y/o
		formas de vida poco saludables, mayor estrés de la población, y hasta exposición a fuentes de radiación artificiales.

		A pesar de todo, el cuerpo tiene una serie de mecanismos para prevenir la aparición y la proliferación de esta enfermedad. Estos,
		son muchos y muy diversos, así como, los mecanismos propios de las células que intervienen en este proceso. En esta simulación, por tanto,
		se pretenden tener en cuenta el enfoque de los autores del trabajo original, eligiendo sólo un subconjunto de todas estas
		carácteristicas y comportamientos.

		La elección de un autómata celular como enfoque para modelizar el comportamiento de esta enfermedad proviene por la relativa
		facilidad de obtener comportamientos complejos a partir de una reglas relativamente sencillas a nivel local en cada célula.

		Este trabajo, como se ha comentado previamente, pretende reproducir el trabajo de José Santos y Ángel Monteagudo de Julio de 2014,
		uno de los muchos trabajos que dichos autores han realizado siguiendo la idea de realizar simulaciones sobre esta enfermedad y, por ello, aquí
		se realiza una nueva implementación con el fin de reproducir los resultados obtenidos en el citado artículo.

		La implementación consiste en seguir un modelo de eventos, es decir, se pretende simular el proceso que siguen las células para su división, que
		consiste en, la realización de una serie de pruebas para comprobar si la célula debe reproducirse, la realización de la propia división y,
		la programación de un nuevo evento de división en el futuro. A lo largo de la ejecución del programa en cada iteración puede haber ninguna, una o
		varias células que deben seguir el proceso que se acaba de describir brevemente.

		Finalmente, para estudiar el comportamiento del cáncer se realizan diversas ejecuciones utilizando diferentes parámetros y configuraciones,
		y observando en cada caso qué factor es el que provoca el cambio en el comportamiento global.
