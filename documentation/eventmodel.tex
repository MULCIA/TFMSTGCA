Los autores, José Santos y Ángel Monteagudo, decidieron utilizar un autómata celular
siguiendo un modelo de eventos. Es decir, por un lado, la simulación se realiza
sobre una rejilla en tres dimensiones, comenzando con una única célula en el centro de la misma.
Por otro lado, se programan una serie de eventos para cada una de las células de la rejilla, aleatoriamente
entre 5 y 10 (ambos inclusive) iteraciones en el futuro.

Las características y propiedades del sistema se describen en las secciones de este capítulo que
se muestran a continuación.

\section{Las células, genoma y propiedades}

Cada célula estará alojada en una única posición del autómata. En esta simulación, no se modeliza
el tamaño de las células, es decir, aunque en las células cancerosas se observa, además de un comportamiento
replicativo sin control, un crecimiento en su tamaño sin control, los autores no han tenido en cuenta esto.

El genoma de cada célula presente en la rejilla cuenta con un genoma y unas propiedades únicas para cada una de ellas.
En cuanto a su genoma, cuenta con 5 variables binarias que representan la presencia o no de una mutación, la cual, define
un determinado comportamiento canceroso. Estas mutaciones son las siguientes:


Además, cada célula tiene una tasa de mutación base y un tamaño de telómero...

\section{Parámetros globales}

\section{Pruebas previas a la mitosis}

\section{Equivalencia temporal}

\section{Bucle principal del modelo de eventos}
