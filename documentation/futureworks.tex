Este trabajo presenta varias de limitaciones que permiten
varias vías de trabajo futuro.

En este trabajo se modelan algunos de los mecanismos que permiten
el crecimiento de tumores, así como, su proliferación. Existen más
marcadores asociados a mutaciones que derivan en un comportamiento cancerosos
de las células que lo contienen que no se tienen en cuenta en este
trabajo. Dos de ellos son el marcador que permite una angiogénesis
sostenida ($AG$) y el marcador que permita la metástasis ($MT$).

Tampoco se tiene en cuenta las células madres del cáncer, que son
aquellas células que si tras aplicar una terapia no son completamente
eliminadas, provocan la reaparición del tumor, llegando a ser más
invasivas que el tumor original en algunos casos.

Otra posibilidad, es tener en cuenta el tamaño y la densidad
de las células que componen el tumor, permitiendo una representación
más fiel de la realidad.

Por último, se podría permitir activar o desactivar ciertos mecanismos
en su conjunto, para poder simular la aplicación de determinadas terapias contra
esta enfermedad.
