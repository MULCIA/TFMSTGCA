El cáncer es una enfermedad compleja y que requiere de una gran cantidad
de eventualidades para que se dé su aparición y proliferación.

El cuerpo humano cuenta con mecanismos para evitar la aparición de esta enfermedad,
que van desde muerte por daño genético hasta la intervención del propio sistema inmune.
Este trabajo se ha centrado en la reproducción de los experimentos para intentar reproducir
los resultados de los autores del trabajo original \cite{jsantos-amonteagudo-1-2014}.

Desde el punto de vista de los marcadores y los parámetros asociados a la simulación se
han realizado distintas configuraciones a fin de estudiar qué comportamiento tiene el sistema y, por tanto,
cómo crece el tumor. La aparición y proliferación de determinadas mutaciones en las células, así como, en qué lugar ocurren,
provocan determinados comportamientos.

El marcador más destacable es el marcador asociado a la capacidad de las células de generar sus propios mensajes de división celular,
o $SG$, el cual propicia la proliferación por la parte exterior de la rejilla.
Esto es algo observado en todos los tipos de tumores, y es lo que hace que se convierta en una
enfermedad mortal, ya que, provoca que crezca y se expanda provocando compresión en tejidos
adyacentes.

Este comportamiento se explica porque encuentra espacio suficiente, y esto permite, además, pasar con éxito dos pruebas previas a la mitosis:
comprobación del factor de crecimiento, y comprobación de ignorancia de inhibición de crecimiento.
Es decir, obtiene sus propios mensajes para realizar la división celular, y no necesita
matar a un vecino para poder realizar la mitosis.

El resto de marcadores, excepto el marcador de inestabilidad genética o $GI$, provocan
que la proliferación e invasividad sea mayor, o lo que es lo mismo, que estemos ante una
masa anormal maligna, o también llamado carcinoma.

El marcador de inestabilidad genética o $GI$ provoca que existan células más propensas
a sufrir mutaciones en el proceso de división celular. Esto conlleva la aparición
de células cancerosas en la parte central de la rejilla, que combinado a otros marcadores
termina por encontrar el camino a su proliferación. Esto, requiere de mayor tiempo de simulación
ya que en el centro de la rejilla hay menos espacio y actúan mecanismos de protección contra
el cáncer de manera más efectiva.

Si esta simulación fuese completa y fiel a la realidad de cualquier tipo de cancer se
podría utilizar como herramienta de capacidad predictiva y, también, como herramienta
para probar la acción de fármacos mediante la inhibición de varios de los mecanismos
que forman la simulación.
